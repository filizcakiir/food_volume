\section{GİRİŞ}

Günümüzde obezite ve aşırı kilo, küresel ölçekte halk sağlığını tehdit eden en önemli sorunlardan biri haline gelmiştir. Dünya Sağlık Örgütü (WHO) 2023 verilerine göre, dünya genelinde 2 milyardan fazla yetişkin aşırı kilolu ve 650 milyonu obez kategorisinde yer almaktadır \cite{who2023obesity}. Türkiye'de ise Sağlık Bakanlığı Türkiye Beslenme ve Sağlık Araştırması (TBSA) 2019 sonuçlarına göre, yetişkin nüfusun \%64.4'ü aşırı kilolu, \%29.9'u ise obez durumundadır \cite{tbsa2019}. Bu veriler, sağlıklı beslenme ve kalori kontrolünün toplum sağlığı açısından kritik önemini ortaya koymaktadır.

Beslenme alışkanlıklarının izlenmesi ve günlük kalori alımının kontrolü, kilo yönetimi ve metabolik hastalıkların önlenmesinde temel stratejilerden biridir \cite{bateman2011process, cordeiro2015barriers}. Ancak, geleneksel el ile kalori takibi yöntemleri, kullanıcılar için zaman alıcı, yorucu ve düşük motivasyona yol açan süreçlerdir. Hall et al. \cite{hall2022calories} tarafından yapılan çalışmada, mobil kalori takibi uygulamalarının kullanıcı devam oranının ilk ay sonunda \%25'in altına düştüğü rapor edilmiştir. Bu düşük devam oranının temel nedenleri arasında, her öğünde yemeklerin el ile aranması, porsiyon ağırlıklarının tahmin edilmesindeki zorluk ve veri girişinin işlemsel yükü yer almaktadır.

Son yıllarda bilgisayarlı görü ve derin öğrenme alanındaki gelişmeler, yemek tanıma sistemlerinin geliştirilmesine olanak sağlamıştır. Bossard et al. \cite{bossard2014food101} tarafından sunulan Food-101 veri kümesi ve bu veri kümesi üzerinde geliştirilen sınıflandırma modelleri, otomatik yemek tanıma literatürünün temelini oluşturmuştur. Benzer şekilde, He et al. \cite{he2016resnet} tarafından önerilen artık ağ mimarisi, görüntü sınıflandırma görevlerinde yüksek başarı göstermiş ve aktarım öğrenmesi yöntemiyle farklı alanlara uyarlanmıştır.

Bu çalışmanın amacı, derin öğrenme tabanlı bir yemek tanıma ve kalori hesaplama sistemi geliştirerek, kullanıcıların mobil aygıtları ile çektikleri tek bir fotoğraftan yemek türünü, porsiyon ağırlığını ve kalori değerini otomatik olarak belirlemelerini sağlamaktır. GastronomGöz olarak adlandırılan sistem, 101 sınıf yemek türünü tanıyabilen ResNet50 tabanlı sınıflandırma modelini, U2NET mimarisi kullanılarak yemeğin arka plandan ayrıştırılmasını \cite{qin2020u2net}, tek görüntüden derinlik tahmini için MiDaS algoritması entegrasyonunu \cite{ranftl2020midas}, bölütleme maskesi ve porsiyon veritabanı kullanılarak ağırlık tahminini, günlük kalori takibi ve kullanıcı profil yönetimini içermektedir. Ayrıca sistem çapraz platform mobil uygulama desteği sunmaktadır. Sistemin geliştirilmesinde Python programlama dili, PyTorch \cite{paszke2019pytorch} ve TensorFlow \cite{abadi2016tensorflow} derin öğrenme çatıları, Flask \cite{grinberg2018flask} ağ çatısı ve Flutter \cite{wu2018flutter} mobil uygulama çatısı kullanılmıştır.

Sağlıklı beslenme ve kilo kontrolü için kalori takibi yapan bireylerin karşılaştığı temel sorunlar arasında el ile veri girişi yükü, porsiyon tahmini hatası, ev yemekleri ve yerel mutfak problemi ve düşük kullanıcı devamlılığı yer almaktadır. MyFitnessPal, Noom ve benzeri popüler uygulamalar, kullanıcıların her öğünde yemeklerini el ile aramasını ve seçmesini gerektirmektedir. Chen et al. \cite{chen2019food} tarafından yapılan kullanılabilirlik çalışmasında, ortalama bir öğün kaydının 5-7 dakika sürdüğü ve bu sürenin kullanıcı motivasyonunu olumsuz etkilediği belirlenmiştir. Amoutzo et al. \cite{amoutzo2021portion} tarafından gerçekleştirilen çalışmada, kullanıcıların porsiyon ağırlıklarını tahmin ederken ortalama \%30-40 oranında hata yaptığı gösterilmiştir. Bu hata, günlük kalori hesaplamalarında ciddi sapmalara yol açmaktadır. Mevcut uygulamaların çoğu, restoran menülerindeki standart yemeklere odaklanmakta ve ev yapımı yemekler ile yerel mutfak kültürüne özgü yemekleri kapsamamaktadır. Cordeiro et al. \cite{cordeiro2015barriers} tarafından 200 katılımcı ile yapılan uzunlamasına çalışmada, kalori takibi uygulamalarının 3 ay içinde \%74 oranında terk edildiği rapor edilmiştir. Bu sorunların çözümü için, yapay zeka destekli otomatik yemek tanıma sistemleri geliştirilmektedir. Min et al. \cite{min2023deep} tarafından yapılan kapsamlı literatür taramasında, derin öğrenme tabanlı yemek tanıma sistemlerinin son 5 yılda önemli oranda arttığı belirlenmiştir.

Yang et al. \cite{yang2010food} tarafından önerilen sistem, SIFT özellikleri ve kelime torbası modeli ile 50 yemek sınıfını \%65 doğrulukla tanımlamıştır. Derin öğrenme çağının başlamasıyla birlikte, evrişimsel sinir ağları yemek tanımada baskın hale gelmiştir. Kagaya et al. \cite{kagaya2014food} tarafından geliştirilen UEC-FOOD 100 veri kümesi ve DeepFood çalışması \cite{liu2016deepfood}, ImageNet önceden eğitilmiş modellerinin yemek tanıma alanında aktarım öğrenme ile kullanılabileceğini göstermiştir \cite{deng2009imagenet}. Bu veri kümesi üzerinde yapılan çalışmalarda, ResNet-50, Inception-v3 ve EfficientNet gibi modern evrişimsel sinir ağı mimarileri kullanılarak \%85-90 arasında doğruluk oranları elde edilmiştir \cite{he2016resnet, szegedy2016inceptionv3, tan2019efficientnet}.

Yemek tanıma sistemlerinde, yemeğin arka plandan ayrıştırılması (foreground-background segmentation) porsiyon tahmini için kritik öneme sahiptir. Geleneksel segmentasyon yöntemleri (GrabCut \cite{rother2004grabcut}, Watershed) manuel kullanıcı müdahalesi gerektirmektedir.

Derin öğrenme tabanlı anlambilimsel bölütleme alanında, Long et al. \cite{long2015fcn} tarafından önerilen tam evrişimsel ağlar mimarisi öncü çalışmadır. Ronneberger et al. \cite{ronneberger2015unet} tarafından geliştirilen U-Net mimarisi, kodlayıcı-çözücü yapısıyla tıbbi görüntü bölütlemede yüksek başarım göstermiştir.

Qin et al. \cite{qin2020u2net} tarafından önerilen U2-Net mimarisi, iç içe U-yapıları ile daha derin öznitelik öğrenme kapasitesine sahiptir ve görüntü belirginlik tespiti görevinde en ileri sonuçlar elde etmiştir. U2NET-P hafif sürümü, 4.7M parametre ile mobil ve gömülü sistemler için uygun hale getirilmiştir.

Yemek görüntülerinden ağırlık tahmini, yemek tanıma sistemlerinin zorlayıcı alt problemlerinden biridir. Puri et al. \cite{puri2009automated} tarafından önerilen sistemde, sahneye bilinen boyutta bir referans nesne yerleştirilerek ölçekleme yapılmaktadır. Bu yöntem, kullanıcıdan ekstra çaba gerektirmesi nedeniyle pratik uygulamalarda sınırlı kullanıma sahiptir. Tek kamera görüntüsünden derinlik bilgisi çıkarmak için Eigen et al. \cite{eigen2014depth} ve Ranftl et al. \cite{ranftl2020midas} tarafından önerilen derin öğrenme modelleri kullanılmaktadır. MiDaS modeli, çok ölçekli görüntüler üzerinde eğitilerek sağlamlık kazanmıştır. Fang et al. \cite{fang2018nutrition5k} tarafından önerilen Nutrition5k veri kümesi, her yemek için gerçek ağırlık ve hacim bilgisi içermektedir. Bu veri kümesi kullanılarak, yemek sınıfı ve bölütleme maskesi alanına dayalı gerileme modelleri eğitilmektedir.

Mobil aygıtlarda çalışan sağlık uygulamaları, kullanıcıların günlük yaşamlarında sağlık verilerini takip etmelerini kolaylaştırmaktadır. Firebase \cite{moroney2017firebase} ve AWS Amplify gibi bulut arka uç hizmetleri, mobil sağlık uygulamalarının hızlı geliştirilebilmesini sağlamaktadır. Flutter çatısı \cite{wu2018flutter}, Dart programlama dili kullanarak tek kod tabanı ile iOS ve Android platformlarında çalışabilen uygulamalar geliştirme olanağı sunmaktadır. Biessek et al. \cite{biessek2020performance} tarafından yapılan başarım karşılaştırmasında, Flutter'ın yerel uygulamalara yakın başarım gösterdiği rapor edilmiştir.

Piyasada bulunan AI destekli kalori takibi uygulamaları arasında Calorie Mama, Foodvisor, Bitesnap ve Lose It! yer almaktadır. Bu uygulamaların karşılaştırmalı analizi Tablo \ref{tab:existing_systems}'de sunulmuştur.

\begin{table}[h]
\centering
\caption{Mevcut AI Destekli Kalori Takibi Uygulamalarının Karşılaştırması}
\label{tab:existing_systems}
\begin{tabular}{|l|c|c|c|c|}
\hline
\textbf{Özellik} & \textbf{Calorie Mama} & \textbf{Foodvisor} & \textbf{Bitesnap} & \textbf{GastronomGöz} \\ \hline
AI Yemek Tanıma & \checkmark & \checkmark & \checkmark & \checkmark \\ \hline
Sınıf Sayısı & 1000+ & 1200+ & 500+ & 101 \\ \hline
Ağırlık Tahmini & \checkmark & \checkmark & $\times$ & \checkmark \\ \hline
Segmentasyon & $\times$ & \checkmark & $\times$ & \checkmark \\ \hline
Türkçe Dil Desteği & $\times$ & $\times$ & $\times$ & \checkmark \\ \hline
Açık Kaynak & $\times$ & $\times$ & $\times$ & Kısmi \\ \hline
Ücretlendirme & \$9.99/ay & \$12.99/ay & \$8.99/ay & Ücretsiz \\ \hline
\end{tabular}
\end{table}

Mevcut sistemlerin temel eksiklikleri yerel ve bölgesel yemeklerin tanınmasında yetersizlik, yüksek aylık abonelik ücretleri, bölütleme kalitesinin düşük olması, çevrimdışı mod desteğinin bulunmaması ve kullanıcı geri bildirim mekanizmalarının yetersizliği olarak özetlenebilir.

Bu tez çalışmasının bilimsel ve teknolojik katkıları şunlardır: Yemek sınıflandırma, bölütleme ve derinlik tahmini modellerinin bütünleştirildiği bir yapay zeka işlem hattı geliştirilmiştir. Bölütleme maskesi alanı ve yemek sınıfına özgü porsiyon veritabanı kullanılarak, referans nesneye gerek duymayan bir ağırlık tahmini yöntemi önerilmiştir. Mobil ortamlarda sınırlı kaynaklarda çalışabilmek için, modellerin gerektiğinde yüklenmesini sağlayan ve tekil nesne örüntüsü ile bellek verimliliği sağlayan bir mimari tasarlanmıştır. Arka uç programlama arayüzü, veritabanı yönetimi, ağ belirteci kimlik doğrulaması ve çapraz platform mobil uygulaması içeren eksiksiz bir sistem gerçekleştirilmiştir. Türk kullanıcılar için yerelleştirilmiş arayüz ve Türkçe yemek adları desteği sağlanmıştır.

\pagebreak{}
