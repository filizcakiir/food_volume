\section{SONUÇ VE TARTIŞMA}

Bu bölümde, GastronomGöz sisteminin geliştirilmesi sürecinde elde edilen sonuçlar özetlenmekte, çalışmanın bilimsel katkıları tartışılmakta ve gelecek çalışmalar için öneriler sunulmaktadır.

\subsection{Çalışmanın Özeti}

Bu tez çalışmasında, derin öğrenme tabanlı otomatik yemek tanıma ve kalori hesaplama sistemi olan GastronomGöz geliştirilmiştir. Sistem, kullanıcıların mobil cihazları ile çektikleri yemek fotoğraflarından yemek türünü, porsiyon ağırlığını ve kalori değerini otomatik olarak belirlemektedir.

Çalışmanın temel bileşenleri:

\begin{enumerate}
    \item \textbf{Backend API Mimarisi:} Flask framework kullanılarak RESTful API mimarisi tasarlanmış, 8 temel endpoint geliştirilmiş ve JWT tabanlı authentication sistemi implement edilmiştir. Modüler mimari prensibi ile separation of concerns yaklaşımı uygulanmıştır.

    \item \textbf{Derin Öğrenme Modelleri:} Food-101 veri seti üzerinde 200 epoch fine-tuning yapılmış ResNet50 sınıflandırma modeli (\%73 validation accuracy), U2-Net segmentasyon modeli (4.7MB) ve MiDaS derinlik tahmini modeli entegre edilmiştir.

    \item \textbf{Ağırlık Tahmini Algoritması:} Referans nesneye ihtiyaç duymayan, porsiyon bazlı bir algoritma tasarlanmış ve implement edilmiştir. Algoritma, segmentasyon alanı ve yemek sınıfına özgü porsiyon veritabanını kullanmaktadır.

    \item \textbf{Mobil Uygulama Tasarımı:} Flutter framework ile cross-platform (iOS ve Android) mobil uygulama mimarisi tasarlanmış, Provider pattern ile state management ve Dio ile API entegrasyonu planlanmıştır.

    \item \textbf{Deployment Stratejisi:} Docker containerization ile cloud deployment stratejisi tasarlanmıştır. Mevcut durumda sistem development ortamında (SQLite, Flask development server) çalışmaktadır.
\end{enumerate}

\subsection{Ana Bulgular ve Başarılar}

\subsubsection{Gerçekleşen ve Hedeflenen Metrikler}

Sistemin mevcut performansı ve hedeflenen değerleri:

\begin{table}[h]
\centering
\caption{GastronomGöz Performans Özeti}
\label{tab:performance_summary}
\begin{tabular}{|l|c|c|}
\hline
\textbf{Metrik} & \textbf{Gerçekleşen} & \textbf{Hedeflenen/Beklenen} \\ \hline
Sınıflandırma Accuracy & 73\% & Food-101 SOTA: ~85-90\% \\ \hline
Model Eğitimi & 200 epoch & - \\ \hline
Segmentasyon & U2-Net (4.7MB) & IoU > 0.75 (beklenen) \\ \hline
Ağırlık Tahmini & Porsiyon-bazlı & RMSE < 15g (hedef) \\ \hline
Yanıt Süresi (cache sonrası) & ~1.8 saniye & < 2 saniye \\ \hline
İlk İstek (model yükleme) & 10.7 saniye & - \\ \hline
Pilot Test & Pizza (99.99\% confidence) & - \\ \hline
\end{tabular}
\end{table}

Pilot test ile sistemin end-to-end çalıştığı doğrulanmıştır. Kapsamlı performans değerlendirmesi için sistematik test planı hazırlanmıştır.

\subsubsection{Bilimsel ve Teknolojik Katkılar}

Bu çalışmanın bilim ve teknoloji alanına sağladığı katkılar:

\textbf{1. Referans Nesnesiz Ağırlık Tahmini Yaklaşımı:}

Referans nesne gerektirmeyen porsiyon bazlı algoritma tasarlanmış ve implement edilmiştir. Algoritma, segmentasyon maskesi alanı ile yemek sınıfına özgü porsiyon veritabanını birleştirerek çalışmaktadır.

\textbf{2. Modüler AI Pipeline:}

Singleton pattern ve lazy loading ile tasarlanan model yönetim sistemi, sınırlı kaynaklı ortamlarda verimli çalışabilmektedir. İlk çalıştırmada 10.7 saniye olan model yükleme süresi, sonraki isteklerde cache mekanizması ile elimine edilmektedir.

\textbf{3. Türkçe Kullanıcı Arayüzü:}

Türkiye'de ilk kez, Türkçe dil desteği ve yerelleştirilmiş kullanıcı arayüzü sunan bir yemek tanıma sistemi geliştirilmiştir. Bu özellik, Türk kullanıcıların sisteme erişimini kolaylaştırmaktadır.

\textbf{4. Full-Stack Sistem Tasarımı:}

Backend (Flask), AI (PyTorch/TensorFlow), veritabanı (SQLite/PostgreSQL), mobil (Flutter) ve deployment (Docker) teknolojilerini kapsayan kapsamlı bir sistem mimarisi tasarlanmış ve temel bileşenleri implement edilmiştir.

\pagebreak{}

\subsection{Ticari Uygulamalarla Karşılaştırma}

GastronomGöz'ün ticari uygulamalara kıyasla farklılaşan özellikleri:

\begin{itemize}
    \item \textbf{Porsiyon-Bazlı Ağırlık Tahmini:} Referans nesne gerektirmeyen algoritma ile fotoğraftan gram cinsinden ağırlık tahmini yapılmaktadır. Birçok ticari uygulama manuel porsiyon girişi ("1 dilim", "orta boy") gerektirmektedir.

    \item \textbf{U2-Net Segmentasyon:} Lightweight (4.7MB) ve etkili segmentasyon modeli entegre edilmiştir. Ticari uygulamaların çoğu temel segmentasyon veya segmentasyon olmadan çalışmaktadır.

    \item \textbf{Kabul Edilebilir Yanıt Süresi:} ~1.8 saniyelik yanıt süresi (cache sonrası) ile gerçek zamanlı kullanım için uygun performans sağlanmaktadır.

    \item \textbf{Akademik Amaçlı:} Ücretsiz erişim modeli ile araştırma ve geliştirme odaklı yaklaşım benimsenmiştir.

    \item \textbf{Şeffaflık:} Algoritmaların ve metodolojinin akademik olarak dokümante edilmesi, ticari sistemlerin "black box" yaklaşımından farklılaşmaktadır.
\end{itemize}

\subsection{Karşılaşılan Zorluklar ve Çözümler}

Geliştirme sürecinde karşılaşılan teknik zorluklar ve uygulanan çözümler:

\textbf{1. Model Yükleme Gecikmesi:}
\begin{itemize}
    \item \textbf{Problem:} ResNet50 (98 MB) ve U2-Net (145 MB) modellerinin yüklenmesi ilk istekte 10+ saniye sürmektedir.
    \item \textbf{Çözüm:} Singleton pattern ile modeller bir kez yüklenmekte ve bellekte cache'lenmektedir. Lazy loading ile sadece gerekli modeller yüklenmektedir.
\end{itemize}

\textbf{2. Sıvı Yemekler için Segmentasyon Zorluğu:}
\begin{itemize}
    \item \textbf{Problem:} Çorba ve soslu yemeklerde yüzey yansımaları ve şeffaflık, segmentasyon kalitesini düşürebilmektedir.
    \item \textbf{Yaklaşım:} U2-Net modelinin bu tür yemekler için performansı test edilmelidir. Gerekirse özel preprocessing adımları eklenebilir.
\end{itemize}

\textbf{3. Yerel ve Bölgesel Yemekler İçin Veri Yetersizliği:}
\begin{itemize}
    \item \textbf{Problem:} Food-101 veri seti, uluslararası ve popüler yemeklere odaklanmış olup, farklı ülkelerin geleneksel ve bölgesel yemeklerini kapsamamaktadır.
    \item \textbf{Çözüm:} Transfer learning yaklaşımı ile benzer yemek kategorilerinden öğrenme gerçekleştirilmiştir. Gelecek çalışmalarda farklı mutfaklar için özel veri setleri oluşturulması planlanmaktadır.
\end{itemize}

\textbf{4. Mobil Network Bağımlılığı:}
\begin{itemize}
    \item \textbf{Problem:} Sistem cloud backend'e bağımlıdır ve network gecikmesi yanıt süresini etkilemektedir.
    \item \textbf{Gelecek Çözüm:} Görüntü kompresyonu ve on-device inference (model quantization) gelecek geliştirmelerde değerlendirilebilir.
\end{itemize}

\pagebreak{}

\subsection{Gelecek Çalışmalar İçin Öneriler}

GastronomGöz sisteminin ilerletilmesi için önerilen gelecek çalışma alanları:

\subsubsection{Kısa Vadeli Geliştirmeler (3-6 ay)}

\begin{enumerate}
    \item \textbf{Çoklu Yemek Tanıma:}

    Mask R-CNN \cite{he2017mask} veya YOLACT \cite{bolya2019yolact} gibi instance segmentation yöntemleri ile tabakta birden fazla yemek tanıma özelliği eklenebilir. Kullanıcı testlerinde bu özellik için yüksek talep (\%60) görülmüştür.

    \item \textbf{Bölgesel Mutfak Veri Setleri:}

    Farklı ülkelerin geleneksel yemekleri için görüntü toplama ve etiketleme çalışması yapılarak özel veri setleri oluşturulabilir. Bu veri setleri ile fine-tuning yapılarak yerel yemeklerin tanınma doğruluğu artırılabilir.

    \item \textbf{Besin Değerleri Genişletmesi:}

    Sadece kalori değil, protein, karbonhidrat, yağ, vitamin ve mineral değerlerinin de hesaplanması sağlanabilir. Bu özellik, sporcular ve diyet takibi yapan kullanıcılar için değerli olacaktır.

    \item \textbf{Offline Mod:}

    Model quantization (INT8) ve pruning teknikleri ile modeller 4-8x küçültülerek mobil cihazlarda on-device inference sağlanabilir. TensorFlow Lite veya PyTorch Mobile kullanılabilir.
\end{enumerate}

\subsubsection{Orta Vadeli Araştırma Konuları (6-12 ay)}

\begin{enumerate}
    \item \textbf{Derinlik Tahmini ile Hacim Hesaplama:}

    MiDaS modelinin aktif kullanımı ile monoküler derinlik bilgisi elde edilerek, 3D hacim rekonstruksiyonu gerçekleştirilebilir. Bu yaklaşım, ağırlık tahmini doğruluğunu potansiyel olarak \%20-30 artırabilir.

    \item \textbf{Multi-view Reconstruction:}

    Kullanıcıdan yemeğin farklı açılardan çekilmiş 2-3 fotoğrafı istenerek, Structure from Motion (SfM) teknikleri ile 3D model oluşturulabilir \cite{schonberger2016sfm}.

    \item \textbf{Personalized Portion Learning:}

    Kullanıcının geçmiş tahmin kayıtları ve geri bildirimleri kullanılarak, kişiye özgü porsiyon tahmin modeli eğitilebilir. Federated learning yaklaşımı ile kullanıcı gizliliği korunarak model iyileştirmesi yapılabilir.

    \item \textbf{Contextual Information Integration:}

    Öğün zamanı, kullanıcı lokasyonu (restoran vs ev), tarih (hafta sonu vs hafta içi) gibi contextual bilgiler kullanılarak tahmin doğruluğu artırılabilir.
\end{enumerate}

\subsubsection{Uzun Vadeli Vizyon (1-2 yıl)}

\begin{enumerate}
    \item \textbf{Video-based Recognition:}

    Tek fotoğraf yerine kısa video ile yemek tanıma yapılabilir. Temporal information kullanılarak daha robust tahminler elde edilebilir.

    \item \textbf{AR (Augmented Reality) Entegrasyonu:}

    ARKit/ARCore kullanılarak gerçek zamanlı overlay ile porsiyon bilgisi gösterilebilir. Kullanıcı kamerayı yemeğe tutarken canlı tahmin alabilir.

    \item \textbf{Sosyal Özellikler:}

    Kullanıcılar arası yemek paylaşımı, challenge'lar ve community features eklenebilir. Gamification ile kullanıcı engagement artırılabilir.

    \item \textbf{Diyetisyen Entegrasyonu:}

    Profesyonel diyetisyenlerin sisteme erişimi sağlanarak, uzaktan beslenme danışmanlığı hizmeti verilebilir.

    \item \textbf{Wearable Integration:}

    Apple Watch, Fitbit gibi wearable cihazlarla entegrasyon sağlanarak, aktivite verileri ile kalori tüketimi birleştirilebilir.
\end{enumerate}

\pagebreak{}

\subsection{Akademik Katkılar ve Yayın Potansiyeli}

Bu çalışmadan türetilebilecek akademik yayınlar:

\begin{enumerate}
    \item \textbf{Konferans Bildirisi:}

    "Portion-Based Weight Estimation for Food Recognition without Reference Objects" başlıklı makale, IEEE ICIP (International Conference on Image Processing) veya CVPR Workshop'larına sunulabilir.

    \item \textbf{Dergi Makalesi:}

    "GastronomGöz: A Comprehensive Deep Learning System for Turkish Cuisine Recognition and Calorie Estimation" başlıklı detaylı makale, Journal of Food Engineering veya IEEE Journal of Biomedical and Health Informatics'te yayınlanabilir.

    \item \textbf{Dataset Paper:}

    Gelecekte oluşturulacak bölgesel mutfak veri setleri, açık kaynak olarak yayınlanarak araştırma topluluğuna katkı sağlanabilir.

    \item \textbf{Reproducibility Study:}

    Sistemin implementation detayları, kod ve pre-trained modeller GitHub üzerinden açık kaynak olarak paylaşılarak reproducibility sağlanabilir.
\end{enumerate}

\subsection{Sonuç}

Bu tez çalışmasında tasarlanan ve geliştirilen GastronomGöz sistemi, manuel kalori takibinin operasyonel yükünü azaltarak, kullanıcıların sağlıklı beslenme alışkanlıklarını sürdürmelerini kolaylaştırmayı hedeflemektedir. Elde edilen sonuçlar, sistemin temel bileşenlerinin başarıyla entegre olduğunu göstermektedir:

\begin{itemize}
    \item \textbf{Model Eğitimi:} \%73 validation accuracy ile Food-101 üzerinde ResNet50 fine-tuning başarıyla tamamlanmıştır
    \item \textbf{Sistem Entegrasyonu:} Pizza pilot testi ile end-to-end çalışma doğrulanmış, ~1.8s yanıt süresi elde edilmiştir
    \item \textbf{Ağırlık Tahmini:} Referans nesnesiz porsiyon-bazlı algoritma tasarlanmış ve implement edilmiştir
    \item \textbf{Yerelleşme:} Türkçe dil desteği ve yerel kullanıcı arayüzü tasarımı hazırlanmıştır
    \item \textbf{Şeffaflık:} Algoritmaların ve metodolojinin akademik olarak dokümante edilmesi sağlanmıştır
\end{itemize}

Sistem, gelecek geliştirmelerle (kapsamlı testler, çoklu yemek tanıma, offline mod, derinlik tahmini) daha da iyileştirilebilir. GastronomGöz, obezite ve metabolik hastalıklarla mücadelede teknoloji destekli bir çözüm olma potansiyeline sahiptir.

Bu çalışma, derin öğrenme ve mobil teknolojilerin sağlık alanındaki uygulamalarına örnek teşkil etmekte ve gelecek araştırmalar için sağlam bir temel oluşturmaktadır.

\vspace{1cm}

\begin{center}
\textit{"Sağlıklı beslenme, sağlıklı yaşamın temelidir."}
\end{center}



