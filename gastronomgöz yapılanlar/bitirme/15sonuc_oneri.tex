\section{SONUÇ VE TARTIŞMA}

Bu tez çalışmasında, derin öğrenme tabanlı otomatik yemek tanıma ve kalori hesaplama sistemi olan GastronomGöz geliştirilmiştir. Sistem, kullanıcıların mobil aygıtları ile çektikleri yemek fotoğraflarından yemek türünü, porsiyon ağırlığını ve kalori değerini otomatik olarak belirlemektedir.

Çalışmanın temel bileşenleri şunlardır: Flask çatısı kullanılarak temsili durum aktarımı programlama arayüzü mimarisi tasarlanmış, 10 temel uç nokta geliştirilmiş ve ağ belirteci tabanlı kimlik doğrulama sistemi gerçekleştirilmiştir. Modüler mimari ilkesi ile kaygıların ayrılması yaklaşımı uygulanmıştır. Food-101 veri kümesi üzerinde 200 dönem ince ayar yapılmış ResNet50 sınıflandırma modeli (\%73 doğrulama başarımı), U2-Net bölütleme modeli (4.7MB) ve MiDaS derinlik tahmini modeli bütünleştirilmiştir. Referans nesneye gerek duymayan, porsiyon bazlı bir algoritma tasarlanmış ve gerçekleştirilmiştir. Algoritma, bölütleme alanı ve yemek sınıfına özgü porsiyon veritabanını kullanmaktadır. Flutter çatısı ile çapraz platform mobil uygulama mimarisi tasarlanmış, sağlayıcı örüntüsü ile durum yönetimi ve Dio ile programlama arayüzü entegrasyonu planlanmıştır. Kapsayıcılaştırma ile bulut dağıtım stratejisi tasarlanmıştır. Mevcut durumda sistem geliştirme ortamında çalışmaktadır.

Sistemin mevcut başarımı ve hedeflenen değerleri Tablo \ref{tab:performance_summary}'de sunulmuştur. Sınıflandırma başarımı \%73 olarak elde edilmiş, bu değer Food-101 veri kümesi üzerinde literatürdeki en ileri sonuçların (\%85-90) altında kalmaktadır. Model eğitimi 200 dönem olarak gerçekleştirilmiştir. Bölütleme için U2-Net modeli (4.7MB) kullanılmış ve beklenen kesişim bölümü birleşim oranı 0.75'in üzerinde hedeflenmiştir. Ağırlık tahmini porsiyon bazlı yaklaşımla tasarlanmış ve hedef ortalama kare kök hata değeri 15 gramın altında belirlenmiştir. Yanıt süresi önbellekleme sonrası yaklaşık 1.8 saniye olarak ölçülmüş, bu değer 2 saniye hedefini karşılamaktadır. İlk istek model yükleme süresi 10.7 saniye olarak kaydedilmiştir. Pilot test kapsamında pizza görüntüsü \%99.99 güven değeri ile doğru sınıflandırılmıştır.

\begin{table}[h]
\centering
\caption{GastronomGöz Performans Özeti}
\label{tab:performance_summary}
\begin{tabular}{|l|c|c|}
\hline
\textbf{Metrik} & \textbf{Gerçekleşen} & \textbf{Hedeflenen/Beklenen} \\ \hline
Sınıflandırma Başarımı & 73\% & Food-101 SOTA: ~85-90\% \\ \hline
Model Eğitimi & 200 dönem & - \\ \hline
Bölütleme & U2-Net (4.7MB) & IoU > 0.75 (beklenen) \\ \hline
Ağırlık Tahmini & Porsiyon-bazlı & RMSE < 15g (hedef) \\ \hline
Yanıt Süresi (önbellek sonrası) & ~1.8 saniye & < 2 saniye \\ \hline
İlk İstek (model yükleme) & 10.7 saniye & - \\ \hline
Pilot Test & Pizza (\%99.99 güven) & - \\ \hline
\end{tabular}
\end{table}

Pilot test ile sistemin uçtan uca çalıştığı doğrulanmıştır. Kapsamlı başarım değerlendirmesi için sistematik test planı hazırlanmıştır.

Bu çalışmanın bilim ve teknoloji alanına sağladığı katkılar şunlardır: Referans nesne gerektirmeyen porsiyon bazlı algoritma tasarlanmış ve gerçekleştirilmiştir. Algoritma, bölütleme maskesi alanı ile yemek sınıfına özgü porsiyon veritabanını birleştirerek çalışmaktadır. Tekil nesne örüntüsü ve geç yükleme ile tasarlanan model yönetim sistemi, sınırlı kaynaklı ortamlarda verimli çalışabilmektedir. İlk çalıştırmada 10.7 saniye olan model yükleme süresi, sonraki isteklerde önbellek mekanizması ile ortadan kalkmaktadır. Türkiye'de Türkçe dil desteği ve yerelleştirilmiş kullanıcı arayüzü sunan bir yemek tanıma sistemi geliştirilmiştir. Bu özellik, Türk kullanıcıların sisteme erişimini kolaylaştırmaktadır. Arka uç, yapay zeka, veritabanı, mobil ve dağıtım teknolojilerini kapsayan kapsamlı bir sistem mimarisi tasarlanmış ve temel bileşenleri gerçekleştirilmiştir.

\pagebreak{}

GastronomGöz'ün ticari uygulamalara kıyasla farklılaşan özellikleri şunlardır: Referans nesne gerektirmeyen algoritma ile fotoğraftan gram cinsinden ağırlık tahmini yapılmaktadır. Birçok ticari uygulama el ile porsiyon girişi gerektirmektedir. Hafif (4.7MB) ve etkili bölütleme modeli bütünleştirilmiştir. Ticari uygulamaların çoğu temel bölütleme veya bölütleme olmadan çalışmaktadır. Yaklaşık 1.8 saniyelik yanıt süresi (önbellekleme sonrası) ile gerçek zamanlı kullanım için uygun başarım sağlanmaktadır. Ücretsiz erişim modeli ile araştırma ve geliştirme odaklı yaklaşım benimsenmiştir. Algoritmaların ve metodolojinin akademik olarak belgelenmesi, ticari sistemlerin kara kutu yaklaşımından farklılaşmaktadır.

Geliştirme sürecinde karşılaşılan teknik zorluklar ve uygulanan çözümler şunlardır: ResNet50 (98 MB) ve U2-Net (145 MB) modellerinin yüklenmesi ilk istekte 10 saniyeden fazla sürmektedir. Tekil nesne örüntüsü ile modeller bir kez yüklenmekte ve bellekte önbelleklenmektedir. Geç yükleme ile sadece gerekli modeller yüklenmektedir. Çorba ve soslu yemeklerde yüzey yansımaları ve şeffaflık, bölütleme kalitesini düşürebilmektedir. U2-Net modelinin bu tür yemekler için başarımı test edilmelidir. Gerekirse özel ön işleme adımları eklenebilir. Food-101 veri kümesi, uluslararası ve popüler yemeklere odaklanmış olup, farklı ülkelerin geleneksel ve bölgesel yemeklerini kapsamamaktadır. Aktarım öğrenmesi yaklaşımı ile benzer yemek kategorilerinden öğrenme gerçekleştirilmiştir. Gelecek çalışmalarda farklı mutfaklar için özel veri kümeleri oluşturulması planlanmaktadır. Sistem bulut arka uca bağımlıdır ve ağ gecikmesi yanıt süresini etkilemektedir. Görüntü sıkıştırması ve aygıt üzerinde çıkarım gelecek geliştirmelerde değerlendirilebilir.

\pagebreak{}

GastronomGöz sisteminin ilerletilmesi için önerilen gelecek çalışma alanları şunlardır: Kısa vadede (3-6 ay), Mask R-CNN veya YOLACT gibi örnek bölütleme yöntemleri ile tabakta birden fazla yemek tanıma özelliği eklenebilir \cite{he2017mask, bolya2019yolact}. Farklı ülkelerin geleneksel yemekleri için görüntü toplama ve etiketleme çalışması yapılarak özel veri kümeleri oluşturulabilir. Sadece kalori değil, protein, karbonhidrat, yağ, vitamin ve mineral değerlerinin de hesaplanması sağlanabilir. Model nicemleme ve budama teknikleri ile modeller 4-8 kat küçültülerek mobil aygıtlarda aygıt üzerinde çıkarım sağlanabilir.

Orta vadede (6-12 ay), MiDaS modelinin aktif kullanımı ile tek görüntüden derinlik bilgisi elde edilerek üç boyutlu hacim yeniden oluşturumu gerçekleştirilebilir. Bu yaklaşım, ağırlık tahmini doğruluğunu potansiyel olarak artırabilir. Kullanıcıdan yemeğin farklı açılardan çekilmiş 2-3 fotoğrafı istenerek, hareket yapılarından modelleme teknikleri ile üç boyutlu model oluşturulabilir \cite{schonberger2016sfm}. Kullanıcının geçmiş tahmin kayıtları ve geri bildirimleri kullanılarak, kişiye özgü porsiyon tahmin modeli eğitilebilir. Öğün zamanı, kullanıcı konumu ve tarih gibi bağlamsal bilgiler kullanılarak tahmin doğruluğu artırılabilir.

Uzun vadede (1-2 yıl), tek fotoğraf yerine kısa video ile yemek tanıma yapılabilir. Artırılmış gerçeklik teknolojileri kullanılarak gerçek zamanlı porsiyon bilgisi gösterilebilir. Kullanıcılar arası yemek paylaşımı ve sosyal özellikler eklenebilir. Profesyonel diyetisyenlerin sisteme erişimi sağlanarak uzaktan beslenme danışmanlığı hizmeti verilebilir. Giyilebilir aygıtlarla entegrasyon sağlanarak aktivite verileri ile kalori tüketimi birleştirilebilir.

Bu çalışmadan türetilebilecek akademik yayınlar arasında referans nesne olmadan porsiyon bazlı ağırlık tahmini konulu konferans bildirisi, Türk mutfağı tanıma ve kalori tahmini konulu dergi makalesi, bölgesel mutfak veri kümeleri açık kaynak yayını ve sistemin gerçekleştirme detayları ile yeniden üretilebilirlik çalışması yer almaktadır.

Bu tez çalışmasında tasarlanan ve geliştirilen GastronomGöz sistemi, el ile kalori takibinin işlemsel yükünü azaltarak, kullanıcıların sağlıklı beslenme alışkanlıklarını sürdürmelerini kolaylaştırmayı hedeflemektedir. Elde edilen sonuçlar, sistemin temel bileşenlerinin başarıyla bütünleştiğini göstermektedir. Model eğitimi kapsamında \%73 doğrulama başarımı ile Food-101 üzerinde ResNet50 ince ayarı başarıyla tamamlanmıştır. Pizza pilot testi ile uçtan uca çalışma doğrulanmış ve yaklaşık 1.8 saniye yanıt süresi elde edilmiştir. Referans nesnesiz porsiyon bazlı algoritma tasarlanmış ve gerçekleştirilmiştir. Türkçe dil desteği ve yerel kullanıcı arayüzü tasarımı hazırlanmıştır. Algoritmaların ve metodolojinin akademik olarak belgelenmesi sağlanmıştır.

Sistem, gelecek geliştirmelerle (kapsamlı testler, çoklu yemek tanıma, çevrimdışı mod, derinlik tahmini) daha da iyileştirilebilir. GastronomGöz, obezite ve metabolik hastalıklarla mücadelede teknoloji destekli bir çözüm olma potansiyeline sahiptir.

Bu çalışma, derin öğrenme ve mobil teknolojilerin sağlık alanındaki uygulamalarına örnek teşkil etmekte ve gelecek araştırmalar için sağlam bir temel oluşturmaktadır.

\vspace{1cm}

\begin{center}
\textit{"Sağlıklı beslenme, sağlıklı yaşamın temelidir."}
\end{center}



