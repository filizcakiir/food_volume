\section{BULGULAR VE DENEYSEL SONUÇLAR}

Bu bölümde, GastronomGöz sisteminin performans değerlendirmesi, gerçekleştirilen testler, hedeflenen performans metrikleri ve karşılaştırmalı analizler sunulmaktadır. Sistemin mevcut test sonuçları, literatür tabanlı beklenen performans değerleri ve gelecek değerlendirme planları detaylı olarak sunulmaktadır.

\subsection{Test Ortamı ve Veri Setleri}

\subsubsection{Donanım ve Yazılım Spesifikasyonları}

Sistem testleri, aşağıdaki donanım ve yazılım konfigürasyonlarında gerçekleştirilmiştir:

\textbf{Backend Test Ortamı:}
\begin{itemize}
    \item \textbf{İşlemci:} Apple M1 Pro (8 core CPU, 14 core GPU)
    \item \textbf{Bellek:} 16 GB Unified Memory
    \item \textbf{İşletim Sistemi:} macOS Sonoma 14.0
    \item \textbf{Python Versiyonu:} 3.11.5
    \item \textbf{PyTorch:} 2.3.1 (MPS backend)
    \item \textbf{TensorFlow:} 2.16.1 (Metal acceleration)
\end{itemize}

\textbf{Mobil Test Cihazları:}
\begin{itemize}
    \item \textbf{iOS:} iPhone 13 Pro (iOS 17.1)
    \item \textbf{Android:} Samsung Galaxy S22 (Android 13)
\end{itemize}

\subsubsection{Kullanılan Veri Setleri}

Sistem geliştirilmesi ve model eğitimi için aşağıdaki veri setleri kullanılmıştır:

\begin{table}[h]
\centering
\caption{Eğitim ve Test İçin Kullanılan Veri Setleri}
\label{tab:eval_datasets}
\begin{tabular}{|l|c|c|p{5cm}|}
\hline
\textbf{Veri Seti} & \textbf{Görüntü Sayısı} & \textbf{Sınıf Sayısı} & \textbf{Kullanım Amacı} \\ \hline
Food-101 Training & 75,750 & 101 & Model eğitimi \\ \hline
Food-101 Test & 25,250 & 101 & Model validasyonu \\ \hline
Pilot Test & 1 & 1 & Sistem entegrasyonu testi \\ \hline
\end{tabular}
\end{table}

\textbf{Not:} Kapsamlı performans değerlendirmesi için sistematik test planı hazırlanmıştır ve gelecek çalışmalarda uygulanacaktır.

\subsection{Yemek Sınıflandırma Performansı}

\subsubsection{ResNet50 Model Eğitimi ve Doğruluğu}

Food-101 veri seti üzerinde 200 epoch süreyle fine-tuning yapılmış ResNet50 modelinin performans metrikleri:

\begin{table}[h]
\centering
\caption{Yemek Sınıflandırma Performans Metrikleri}
\label{tab:classification_metrics}
\begin{tabular}{|l|c|}
\hline
\textbf{Metrik} & \textbf{Değer} \\ \hline
Training Accuracy (Final) & 65.0\% \\ \hline
Validation Accuracy (Final) & 73.0\% \\ \hline
Training Loss (Final) & 1.8 \\ \hline
Validation Loss (Final) & 1.5 \\ \hline
Epoch Sayısı & 200 \\ \hline
Model Size & 98.7 MB (191 MB .hdf5) \\ \hline
Sınıf Sayısı & 101 \\ \hline
Framework & Keras/TensorFlow \\ \hline
\end{tabular}
\end{table}

Model, 200 epoch eğitim sonrasında validation seti üzerinde \%73.0 doğruluk oranına ulaşmıştır. Training ve validation accuracy eğrileri, modelin öğrenme sürecinin kararlı bir şekilde ilerlediğini göstermektedir. Food-101 veri seti gibi geniş ve çeşitli sınıflara sahip bir veri seti üzerinde bu performans, transfer learning yaklaşımının etkinliğini göstermektedir \cite{bossard2014food101}.

\pagebreak{}

\textbf{Model Eğitim Süreci:}

Eğitim ve doğrulama eğrileri, modelin öğrenme dinamiklerini ortaya koymaktadır:

\begin{itemize}
    \item \textbf{Eğitim İlerlemesi:} İlk 50 epoch'ta hızlı öğrenme, sonrasında kademeli iyileşme gözlenmiştir
    \item \textbf{Overfitting Kontrolü:} Training ve validation accuracy arasındaki gap (~8\%) kontrol altındadır
    \item \textbf{Loss Azalması:} Loss değerleri düzenli bir şekilde azalmış, convergence sağlanmıştır
    \item \textbf{Stabilite:} Son 50 epoch'ta değerler stabil seyretmiş, ek eğitimin marjinal fayda sağlayacağı görülmüştür
\end{itemize}

\textbf{Model Performans Analizi:}

\%73 validation accuracy, Food-101 gibi challenging bir veri seti için makul bir sonuçtur. Performans iyileştirme potansiyeli için:
\begin{itemize}
    \item Görsel olarak benzer yemek sınıfları (çorbalar, salatalar) en zorlu kategorilerdir
    \item Belirgin şekil ve renk özelliklerine sahip yemekler (pizza, hamburger) daha kolay tanınmaktadır
    \item Data augmentation ve longer training potansiyel iyileştirme noktalarıdır
\end{itemize}

\subsubsection{Transfer Learning Yaklaşımı}

ResNet50 modeli, ImageNet pre-trained ağırlıklar kullanılarak Food-101 veri seti üzerinde fine-tuning yapılmıştır. Transfer learning yaklaşımının sağladığı avantajlar:

\begin{itemize}
    \item \textbf{Hızlı Convergence:} Pre-trained ağırlıklar sayesinde model daha az epoch'ta öğrenme gerçekleştirebilmektedir
    \item \textbf{Feature Extraction:} ImageNet'ten öğrenilen low-level feature'lar (kenarlar, dokular) yemek tanımada kullanılabilmektedir
    \item \textbf{Sınırlı Veri İhtiyacı:} Transfer learning, Food-101 gibi orta ölçekli veri setlerinde etkin sonuçlar vermektedir
\end{itemize}

Bu proje kapsamında, ResNet50'nin tüm katmanları unfreeze edilerek end-to-end fine-tuning yapılmıştır. Alternative freezing stratejileri (örn. ilk katmanları freeze etme) gelecek çalışmalarda denenebilir \cite{pan2010transfer}.

\pagebreak{}

\subsection{Segmentasyon ve Ağırlık Tahmini}

\subsubsection{U2-Net Segmentasyon Modeli}

U2-Net modeli, yemek görüntülerinden arka plan ayrıştırması için entegre edilmiştir. Segmentasyon kalitesi, Intersection over Union (IoU) metriği ile değerlendirilebilir \cite{rezatofighi2019giou}:

\begin{equation}
\text{IoU} = \frac{\text{Area of Overlap}}{\text{Area of Union}} = \frac{TP}{TP + FP + FN}
\end{equation}

\textbf{U2-Net Model Özellikleri:}
\begin{itemize}
    \item \textbf{Mimari:} U2NET-P (lightweight version, 4.7MB)
    \item \textbf{Giriş/Çıkış:} 320×320 RGB → 320×320 binary mask
    \item \textbf{Kullanım:} Yemek kontur tespiti ve porsiyon alanı hesaplama
    \item \textbf{Framework:} PyTorch
\end{itemize}

\textbf{Segmentasyon Performansı:}

U2-Net modeli, literatürde semantik segmentasyon görevlerinde yüksek performans gösterdiği raporlanmıştır \cite{qin2020u2net}. Yemek segmentasyonu için beklenen IoU değerleri:
\begin{itemize}
    \item Belirgin konturlu yemekler (pizza, steak): IoU > 0.85
    \item Heterojen yapılı yemekler (salata, pilav): IoU 0.70-0.85
    \item Sıvı yemekler (çorba): IoU < 0.70
\end{itemize}

Kapsamlı segmentasyon değerlendirmesi için ground-truth mask'leri içeren test seti hazırlanması planlanmaktadır.

\subsubsection{Segmentasyon Zorlukları}

Yemek segmentasyonunda karşılaşılan temel zorluklar:
\begin{enumerate}
    \item \textbf{Arka plan benzerliği:} Beyaz tabak üzerinde açık renkli yemeklerde kontrast düşüklüğü
    \item \textbf{Gölge ve ışıklandırma:} Değişken ışık koşullarının segmentasyon kalitesine etkisi
    \item \textbf{Çoklu yemek tespiti:} Tabakta birden fazla yemek olduğunda instance segmentation ihtiyacı
    \item \textbf{Sıvı yemekler:} Çorba ve soslu yemeklerde yüzey yansımalarından kaynaklanan belirsizlik
\end{enumerate}

Bu zorlukların çözümü için gelecek çalışmalarda data augmentation (lighting variations, background diversity) ve instance segmentation yöntemleri denenebilir.

\pagebreak{}

\subsection{Porsiyon Bazlı Ağırlık Tahmini}

\subsubsection{Algoritma Yapısı ve Çalışma Prensibi}

Geliştirilen porsiyon bazlı ağırlık tahmin algoritması, referans nesne gerektirmeden çalışmaktadır. Algoritmanın çalışma prensibi:

\begin{enumerate}
    \item Segmentasyon maskesinden yemek alanı hesaplanır (piksel sayısı)
    \item Alan normalize edilir ve porsiyon boyutu belirlenir (small/medium/large)
    \item Yemek sınıfına öz gü porsiyon veritabanından baz ağırlık alınır
    \item Alan faktörü ile ±20\% ince ayar yapılır
\end{enumerate}

\textbf{Porsiyon Veritabanı:} 12 farklı yemek türü için 3 porsiyon boyutu tanımlanmıştır (pizza, hamburger, baklava, steak, vb.). Diğer yemekler için default değerler kullanılmaktadır.

\textbf{Algoritma Avantajları:}
\begin{itemize}
    \item Kullanıcıdan ek efor gerektirmez (referans nesne yerleştirme gerekmez)
    \item Hızlı hesaplama (~18ms)
    \item Porsiyon veritabanı kolayca genişletilebilir
\end{itemize}

\subsubsection{Beklenen Performans ve Doğruluk Analizi}

Porsiyon bazlı algoritmanın beklenen doğruluk seviyeleri, yemek tipine göre değişmektedir:

\textbf{Yüksek Doğruluk Beklenen Kategoriler:}
\begin{itemize}
    \item Katı ve standart şekilli yemekler (pizza, hamburger, steak)
    \item Porsiyon veritabanında tanımlı 12 yemek türü
    \item Belirgin kontur ve tek yemek içeren görüntüler
\end{itemize}

\textbf{Düşük Doğruluk Beklenen Durumlar:}
\begin{itemize}
    \item Sıvı ve soslu yemekler (çorbalar, soslu makarnalar)
    \item Veritabanında olmayan yemekler (default değer kullanılır)
    \item Çoklu yemek içeren tabaklar
\end{itemize}

\textbf{Algoritma Geliştime Potansiyeli:}

Gelecek çalışmalarda, ground-truth ağırlık değerleri ile sistematik test yapılarak RMSE ve MAE metrikleri hesaplanabilir. Literatürde benzer porsiyon-bazlı yöntemler ±15-25g RMSE aralığında performans göstermektedir \cite{fang2018nutrition, pouladzadeh2014volume}. Hedef performans: RMSE < 15g.

\subsubsection{Kalori Hesaplama}

Kalori hesaplaması, tahmin edilen ağırlık ve yemek sınıfına özgü kalori veritabanı kullanılarak yapılmaktadır:

\begin{equation}
\text{Toplam Kalori} = \frac{\text{Tahmini Ağırlık (g)}}{100} \times \text{Kalori}_{100g}
\end{equation}

Örnek: Pizza tahmini (120g, 310 kcal/100g) → 372 kcal

Kalori veritabanı, 101 yemek sınıfı için 100g başına kalori değerlerini içermektedir.

\pagebreak{}

\subsection{Sistem Entegrasyon Testi}

\subsubsection{Pizza Testi - End-to-End Sistem Doğrulaması}

Sistemin tüm bileşenlerinin entegrasyonunu doğrulamak için pizza görüntüsü ile pilot test gerçekleştirilmiştir:

\textbf{Test Senaryosu:}
\begin{itemize}
    \item \textbf{Görüntü:} Pizza fotoğrafı
    \item \textbf{İşlem:} Image upload → ResNet50 classification → U2-Net segmentation → Weight estimation → Calorie calculation
    \item \textbf{Ortam:} Development (Apple M1 Pro, macOS)
\end{itemize}

\textbf{Test Sonuçları:}

\begin{table}[h]
\centering
\caption{Pizza Pilot Test Sonuçları}
\label{tab:pizza_test}
\begin{tabular}{|l|c|}
\hline
\textbf{Metrik} & \textbf{Değer} \\ \hline
Predicted Class & Pizza \\ \hline
Confidence & 99.99\% \\ \hline
Estimated Weight & 120 gram \\ \hline
Calculated Calories & 372 kcal \\ \hline
İlk İstek Süresi & 10.7 saniye \\ \hline
Sonraki İstekler & ~1.8 saniye \\ \hline
\end{tabular}
\end{table}

\textbf{Performans Analizi:}

\begin{itemize}
    \item \textbf{İlk İstek (10.7s):} Model yükleme zamanını içerir. Singleton pattern ile modeller bir kez yüklenir ve bellekte cache'lenir.
    \item \textbf{Sonraki İstekler (~1.8s):} Lazy loading sayesinde sadece inference zamanı. Gerçek zamanlı kullanım için kabul edilebilir performans.
    \item \textbf{Yüksek Confidence (99.99\%):} Pizza gibi belirgin görsel özelliklere sahip yemekler için model çok güvenli tahmin üretmektedir.
\end{itemize}

\textbf{Sonuç:} Pilot test, sistemin tüm bileşenlerinin başarıyla entegre olduğunu ve end-to-end akışın çalıştığını doğrulamıştır.

\subsubsection{Bellek ve Kaynak Kullanımı}

Backend uygulamasının bellek kullanımı:

\begin{itemize}
    \item \textbf{Model Dosyaları:} ResNet50 (191 MB), U2-Net (4.7 MB)
    \item \textbf{Runtime Bellek:} Modeller yüklendikten sonra ~600-700 MB
    \item \textbf{Optimizasyon:} Singleton pattern ile tek instance, lazy loading ile ihtiyaç anında yükleme
\end{itemize}

MiDaS modeli implement edilmiş ancak şu an pasif durumdadır (gelecekte hacim hesaplama için aktifleştirilebilir).

\pagebreak{}

\subsection{Literatür ve Mevcut Sistemlerle Karşılaştırma}

\subsubsection{Ticari Uygulamalarla Fonksiyonel Karşılaştırma}

GastronomGöz'ün ticari AI destekli kalori takibi uygulamalarıyla fonksiyonel karşılaştırması:

\begin{table}[h]
\centering
\caption{GastronomGöz vs. Ticari Uygulamalar - Detaylı Karşılaştırma}
\label{tab:detailed_comparison}
\small
\begin{tabular}{|p{3.5cm}|c|c|c|c|}
\hline
\textbf{Özellik} & \textbf{MyFitnessPal} & \textbf{Foodvisor} & \textbf{Calorie Mama} & \textbf{GastronomGöz} \\ \hline
\multicolumn{5}{|c|}{\textbf{AI Özellikleri}} \\ \hline
Yemek tanıma & $\times$ & \checkmark & \checkmark & \checkmark \\ \hline
Ağırlık tahmini & $\times$ & \checkmark & \checkmark & \checkmark \\ \hline
Segmentasyon & $\times$ & Temel & $\times$ & Gelişmiş \\ \hline
Çoklu yemek tanıma & $\times$ & \checkmark & Sınırlı & Planlı \\ \hline
\multicolumn{5}{|c|}{\textbf{Performans}} \\ \hline
Sınıflandırma doğruluğu & - & ~85\% & ~82\% & 73\% \\ \hline
Ağırlık tahmini & - & Var & Var & Porsiyon-bazlı \\ \hline
Yanıt süresi (cache sonrası) & - & 2-3s & 3-4s & ~1.8s \\ \hline
\multicolumn{5}{|c|}{\textbf{Kullanıcı Deneyimi}} \\ \hline
Türkçe dil desteği & Kısmi & $\times$ & $\times$ & \checkmark \\ \hline
Offline mod & $\times$ & $\times$ & $\times$ & Planlı \\ \hline
Ücretsiz plan & Temel & Temel & Temel & Tam \\ \hline
Aylık ücret (premium) & \$9.99 & \$12.99 & \$8.99 & \$0 \\ \hline
\multicolumn{5}{|c|}{\textbf{Teknik}} \\ \hline
Open source & $\times$ & $\times$ & $\times$ & Kısmi \\ \hline
API erişimi & $\times$ & $\times$ & $\times$ & \checkmark \\ \hline
Veri gizliliği & Sunucu & Sunucu & Sunucu & Hibrit \\ \hline
\end{tabular}
\end{table}

\textbf{GastronomGöz'ün Özgün Değeri ve Rekabet Avantajları:}

\begin{enumerate}
    \item \textbf{Fotoğraftan Otomatik Gram Ölçümü:} Piyasadaki uygulamaların çoğu, kullanıcıdan manuel porsiyon girişi talep etmektedir (örn: "1 dilim pizza", "orta boy tabak"). GastronomGöz, segmentasyon ve porsiyon algoritması ile fotoğraftan direkt gram cinsinden ağırlık tahmini yapmaktadır. Bu özellik, kullanıcı deneyimini önemli ölçüde iyileştirmektedir.

    \item \textbf{U2-Net Segmentasyon:} Lightweight U2-Net modeli (4.7MB) ile hızlı ve etkili segmentasyon. Ticari uygulamaların çoğu temel segmentasyon veya segmentasyon olmadan çalışmaktadır.

    \item \textbf{Türkçe Dil Desteği:} Türkçe kullanıcı arayüzü ve yerelleştirilmiş içerik. Mevcut uygulamaların çoğu sadece İngilizce veya sınırlı Türkçe desteği sunmaktadır.

    \item \textbf{Açık Erişim:} Akademik proje olarak ücretsiz erişim modeli ile kullanıcı erişilebilirliği maksimize edilmiştir.

    \item \textbf{Şeffaf Algoritmalar:} Ağırlık tahmini algoritması ve porsiyon veritabanı dokümante edilmiştir. Ticari sistemler genellikle "black box" yaklaşımı kullanmaktadır.
\end{enumerate}

\pagebreak{}

\subsubsection{Akademik Çalışmalarla Karşılaştırma}

GastronomGöz'ün literatürdeki benzer akademik çalışmalarla yaklaşım karşılaştırması:

\begin{table}[h]
\centering
\caption{Akademik Çalışmalarla Yaklaşım Karşılaştırması}
\label{tab:academic_comparison}
\small
\begin{tabular}{|p{3.5cm}|c|p{5cm}|}
\hline
\textbf{Çalışma} & \textbf{Accuracy} & \textbf{Yaklaşım ve Özellikler} \\ \hline
DeepFood \cite{liu2016deepfood} & 77.4\% & Food-101 veri seti, AlexNet tabanlı \\ \hline
NutriNet \cite{mezgec2017nutrinet} & 86.7\% & Besin değeri tahmini, büyük veri seti \\ \hline
Nutrition5k \cite{fang2018nutrition} & 85.2\% & Depth kamera ile hacim tahmini \\ \hline
Pouladzadeh \cite{pouladzadeh2014volume} & 81.5\% & Referans nesne (özel kart) gerektiren \\ \hline
\textbf{GastronomGöz} & \textbf{73\%} & \textbf{ResNet50 transfer learning, porsiyon-bazlı (referans nesne yok)} \\ \hline
\end{tabular}
\end{table}

GastronomGöz, Food-101 gibi challenging veri setinde transfer learning ile makul performans elde etmiştir. Porsiyon-bazlı ağırlık tahmini yaklaşımı, kullanıcıdan ek efor (referans nesne yerleştirme) gerektirmemesi açısından avantajlıdır.

\subsection{Planlanan Kullanıcı Testleri ve Değerlendirme}

\subsubsection{Kullanılabilirlik Test Planı}

Sistemin kullanıcı deneyimini değerlendirmek için kapsamlı kullanılabilirlik test planı hazırlanmıştır:

\textbf{Hedef Katılımcı Profili:}
\begin{itemize}
    \item Yaş aralığı: 18-55
    \item Farklı kalori takibi deneyim seviyeleri (yeni başlayan, orta, deneyimli)
    \item Mevcut uygulamaları kullananlar ve kullanmayanlar
    \item En az 15-20 katılımcı
\end{itemize}

\textbf{Planlanan Test Görevleri:}
\begin{enumerate}
    \item Kayıt olma ve profil oluşturma
    \item Yemek fotoğrafı çekme ve tahmin alma
    \item Sonuçları değerlendirme ve doğruluğu kontrol etme
    \item Farklı yemek türlerinde (katı, sıvı, karışık) test yapma
\end{enumerate}

\textbf{Değerlendirme Metrikleri:}
\begin{itemize}
    \item Görev tamamlama oranı
    \item Görev tamamlama süresi
    \item System Usability Scale (SUS) Skoru
    \item Kullanıcı memnuniyeti anketi
    \item Tahmin doğruluğu algısı (perceived accuracy)
\end{itemize}

\textbf{Beklenen Değerlendirme Alanları:}
\begin{itemize}
    \item Kullanım kolaylığı ve öğrenme eğrisi
    \item Tahmin hızı ve sonuç güvenilirliği
    \item Arayüz tasarımı ve kullanıcı akışı
    \item İyileştirme önerileri ve özellik talepleri
\end{itemize}

\textbf{Not:} Kullanıcı testleri, sistemin tam entegrasyonundan sonra gerçekleştirilecektir. Şu ana kadar pilot düzeyde teknik testler (pizza testi) yapılmıştır.

\pagebreak{}

\subsection{Tartışma}

\subsubsection{Çalışmanın Güçlü Yönleri}

\textbf{1. Modüler ve Ölçeklenebilir Mimari:}

Singleton pattern ve lazy loading kullanımı, sistemin farklı donanım konfigürasyonlarında verimli çalışmasını sağlamıştır. Factory pattern ile farklı ortamlar (development, testing, production) için ayrı konfigürasyonlar oluşturulabilmektedir.

\textbf{2. Referans Nesnesiz Ağırlık Tahmini:}

Porsiyon bazlı algoritma, kullanıcıdan ekstra efor gerektirmeden (referans nesne yerleştirme) ağırlık tahmini yapmaktadır. Pouladzadeh ve ark. \cite{pouladzadeh2014volume} tarafından önerilen referans nesne yaklaşımına kıyasla kullanılabilirlik açısından üstündür.

\textbf{3. Transfer Learning'in Etkin Kullanımı:}

ImageNet pre-trained ResNet50 ağırlıkları kullanılarak, 200 epoch eğitim sonrasında \%73 validation accuracy elde edilmiştir. Transfer learning yaklaşımı, sınırlı hesaplama kaynakları olan araştırma projeleri için etkili bir yöntemdir.

\textbf{4. End-to-End Sistem Entegrasyonu:}

Pizza pilot testi ile tüm bileşenlerin (classification, segmentation, weight estimation) başarıyla entegre olduğu doğrulanmıştır. Sistem, ~1.8 saniye yanıt süresi ile gerçek zamanlı kullanım için uygun performans göstermektedir.

\subsubsection{Kısıtlamalar ve Zorluklar}

\textbf{1. Veri Seti Sınırlılıkları:}

Food-101 veri seti, 101 sınıfla sınırlıdır ve uluslararası popüler yemeklere odaklanmıştır. Farklı ülkelerin geleneksel ve bölgesel yemekleri için eğitim verisi bulunmaması, bu tür yemeklerin tanınma doğruluğunu düşürmektedir. Gelecek çalışmalarda, bölgesel mutfaklar için özel veri setleri oluşturulması planlanmaktadır.

\textbf{2. Sıvı Yemekler için Beklenen Zorluklar:}

Çorba ve soslu yemekler için segmentasyon kalitesinin düşük olması beklenmektedir. Bu yemeklerde yüzey yansımaları ve şeffaflık, segmentasyon kalitesini olumsuz etkileyebilmektedir.

\textbf{3. Derinlik Bilgisinin Eksikliği:}

MiDaS derinlik tahmini modeli entegre edilmiş ancak henüz aktif kullanılmamaktadır. Hacim bazlı ağırlık hesaplaması için derinlik bilgisinin kullanılması, tahmin doğruluğunu artırma potansiyeline sahiptir.

\textbf{4. Çoklu Yemek Tanıma:}

Mevcut sistem, tek yemek tanımaya odaklanmıştır. Tabakta birden fazla yemek olduğunda, sadece en baskın yemek tanınmaktadır. Instance segmentation yaklaşımı \cite{he2017mask} ile çoklu yemek tanıma gelecek çalışmalara dahil edilecektir.

\textbf{5. Ağ Bağımlılığı:}

Sistem, cloud backend'e bağımlıdır ve offline mod desteklememektedir. Mobil cihazlarda on-device AI inference için model quantization ve pruning teknikleri \cite{han2016deep} uygulanması gerekecektir.

\pagebreak{} 