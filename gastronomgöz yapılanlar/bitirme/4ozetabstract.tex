\phantomsection
\addcontentsline{toc}{section}{ÖZET}


\begin{center}
\textbf{\large ÖZET}

\textbf{GASTRONOMGÖZ: AKILLI YEMEK TANIMA VE KALORİ HESAPLAMA SİSTEMİ UYGULAMASI}
\end{center}

Bu çalışmada, derin öğrenme tabanlı yemek tanıma ve kalori hesaplama sistemi olan GastronomGöz geliştirilmiştir. Sistem, kullanıcıların mobil cihazları ile çektikleri yemek fotoğraflarını analiz ederek otomatik sınıflandırma, porsiyon ağırlığı tahmini ve besin değeri hesaplaması yapmaktadır. Yemek sınıflandırması için ResNet50 evrişimsel sinir ağı mimarisi, görüntü segmentasyonu için U2NET modeli ve derinlik tahmini için MiDaS algoritması kullanılmıştır. Backend katmanında Python Flask framework'ü ile RESTful API mimarisi tasarlanmış, mobil uygulama Dart programlama dili ve Flutter framework'ü ile cross-platform olarak geliştirilmiştir. Sistem, JSON Web Token tabanlı kimlik doğrulama, ilişkisel veritabanı yönetimi ve gerçek zamanlı kalori takibi fonksiyonelliklerini içermektedir. Performans değerlendirmesi sonuçları, yemek sınıflandırma görevinde \%87.2 Top-1 doğruluk oranı, segmentasyon kalitesinde 0.847 Intersection over Union (IoU) metriği ve ağırlık tahmininde ortalama ±12 gram Root Mean Square Error (RMSE) değeri göstermiştir. GastronomGöz, manuel kalori takibinin operasyonel yükünü azaltarak obezite ve metabolik hastalıklarla mücadelede bilişim teknolojileri destekli bir çözüm sunmaktadır.

\vspace{2cm}

\textbf{Anahtar Kelimeler:} Derin Öğrenme, Bilgisayarlı Görü, Yemek Tanıma Sistemleri, Kalori Tahmini, Evrişimsel Sinir Ağları, Semantik Segmentasyon, Mobil Sağlık Uygulamaları





%%%%%%%%%%%%%%%%%%%%%%%%%%%%%%%%%%%%%%%%%%%%%%%%%%%%%%%%%%%%%%%%%%%
\newpage
\begin{center}
\textbf{\large ABSTRACT}

\textbf{GASTRONOMGOZ: SMART FOOD RECOGNITION AND CALORIE CALCULATION SYSTEM APPLICATION}
\end{center}

In this study, GastronomGoz, a deep learning-based food recognition and calorie calculation system, has been developed. The system analyzes food photographs captured by users' mobile devices to perform automatic classification, portion weight estimation, and nutritional value calculation. The ResNet50 convolutional neural network architecture for food classification, the U2NET model for image segmentation, and the MiDaS algorithm for depth estimation were employed. The backend layer was architected as a RESTful API using the Python Flask framework, while the mobile application was developed cross-platform using the Dart programming language and Flutter framework. The system incorporates JSON Web Token-based authentication, relational database management, and real-time calorie tracking functionalities. Performance evaluation results demonstrated 87.2\% Top-1 accuracy in food classification tasks, 0.847 Intersection over Union (IoU) metric in segmentation quality, and an average Root Mean Square Error (RMSE) of ±12 grams in weight estimation. GastronomGoz presents an information technology-supported solution for combating obesity and metabolic diseases by reducing the operational burden of manual calorie tracking.

\vspace{2cm}

\textbf{Keywords:} Deep Learning, Computer Vision, Food Recognition Systems, Calorie Estimation, Convolutional Neural Networks, Semantic Segmentation, Mobile Health Applications

\pagebreak{}
