\phantomsection
\addcontentsline{toc}{section}{ÖZET}


\begin{center}
\textbf{\large ÖZET}

\textbf{GASTRONOMGÖZ: AKILLI YEMEK TANIMA VE KALORİ HESAPLAMA SİSTEMİ UYGULAMASI}
\end{center}

Bu çalışmada, derin öğrenme tabanlı yemek tanıma ve kalori hesaplama sistemi olan GastronomGöz geliştirilmiştir. Sistem, kullanıcıların mobil aygıtları ile çektikleri yemek fotoğraflarını çözümleyerek otomatik sınıflandırma, porsiyon ağırlığı tahmini ve besin değeri hesaplaması yapmaktadır. Yemek sınıflandırması için ResNet50 evrişimsel sinir ağı mimarisi, görüntü bölütleme için U2NET modeli ve derinlik tahmini için MiDaS algoritması kullanılmıştır. Arka uç katmanında Python Flask çatısı ile temsili durum aktarımı uygulama programlama arayüzü tasarlanmış, mobil uygulama Dart programlama dili ve Flutter çatısı ile çapraz platform desteği sağlanarak geliştirilmiştir. Sistem, JSON tabanlı ağ belirteci kimlik doğrulama, ilişkisel veritabanı yönetimi ve gerçek zamanlı kalori takibi işlevlerini içermektedir. Deneysel çalışmalarda Food-101 veri kümesi üzerinde 200 dönem eğitim sonucunda yemek sınıflandırma başarımı \%73 olarak elde edilmiş, bölütleme kalitesi öncül çalışmalarla uyumlu bulunmuştur. Ağırlık tahmin yaklaşımı porsiyon tabanlı ve alan ince-ayarlı algoritma kullanılarak tasarlanmıştır. GastronomGöz, el ile kalori kaydının işlemsel yükünü azaltarak obezite ve metabolik hastalıklarla mücadelede bilişim teknolojileri destekli bir çözüm önermektedir.

\vspace{2cm}

\textbf{Anahtar Kelimeler:} Derin Öğrenme, Yemek Tanıma, Kalori Tahmini, Evrişimsel Sinir Ağları, Mobil Sağlık





%%%%%%%%%%%%%%%%%%%%%%%%%%%%%%%%%%%%%%%%%%%%%%%%%%%%%%%%%%%%%%%%%%%
\newpage
\begin{center}
\textbf{\large ABSTRACT}

\textbf{GASTRONOMGOZ: SMART FOOD RECOGNITION AND CALORIE CALCULATION SYSTEM APPLICATION}
\end{center}

In this study, GastronomGoz, a deep learning-based food recognition and calorie calculation system, has been developed. The system analyzes food photographs captured by users' mobile devices to perform automatic classification, portion weight estimation, and nutritional value calculation. The ResNet50 convolutional neural network architecture for food classification, the U2NET model for image segmentation, and the MiDaS algorithm for depth estimation were employed. The backend layer was architected as a RESTful API using the Python Flask framework, while the mobile application was developed cross-platform using the Dart programming language and Flutter framework. The system incorporates JSON Web Token-based authentication, relational database management, and real-time calorie tracking functionalities. Experimental results on Food-101 dataset demonstrated 73\% validation accuracy after 200 training epochs for food classification. The segmentation quality was found consistent with baseline studies. Weight estimation approach was designed using portion-based and area-refined algorithm. GastronomGoz presents an information technology-supported solution for combating obesity and metabolic diseases by reducing the operational burden of manual calorie tracking.

\vspace{2cm}

\textbf{Keywords:} Deep Learning, Food Recognition, Calorie Estimation, Convolutional Neural Networks, Mobile Health

\pagebreak{}
