\phantomsection
\addcontentsline{toc}{section}{ÖZET}


\begin{center}
\textbf{\large ÖZET}

\textbf{MOBİL UYGULAMA ENTEGRASYONLU AKILLI BESLENME TAKİP SİSTEMİ}
\end{center}

Bu çalışmada, derin öğrenme tabanlı yemek tanıma ve kalori hesaplama sistemi olan GastronomGöz geliştirilmiştir. Sistem, kullanıcıların mobil aygıtları ile çektikleri yemek fotoğraflarını çözümleyerek otomatik sınıflandırma, porsiyon ağırlığı tahmini ve besin değeri hesaplaması yapmaktadır. Yemek sınıflandırması için ResNet50 evrişimsel sinir ağı mimarisi, görüntü bölütleme için U²-Net modeli ve derinlik bilgisi desteği için MiDaS algoritması kullanılmıştır. Arka uç katmanında Python Flask çatısı ile temsili durum aktarımı uygulama programlama arayüzü tasarlanmış ve 25 uç nokta gerçekleştirilmiş, mobil uygulama Dart programlama dili ve Flutter çatısı ile iOS ve Android platformları için tam işlevsel olarak geliştirilmiştir. Sistem, JSON tabanlı ağ belirteci kimlik doğrulama, ilişkisel veritabanı yönetimi, gerçek zamanlı kalori takibi, günlük ve haftalık istatistikler, yemek düzenleme ve silme, favori işaretleme, başarı rozetleri ve kullanıcı bildirimleri işlevlerini içermektedir. Deneysel çalışmalarda Food-101 veri kümesi üzerinde 200 epoch eğitim sonucunda yemek sınıflandırma doğrulama başarımı \%73 olarak elde edilmiş, bölütleme kalitesi öncül çalışmalarla uyumlu bulunmuştur. Ağırlık tahmin yaklaşımı porsiyon tabanlı ve alan ince-ayarlı algoritma kullanılarak tasarlanmış, MiDaS derinlik bilgisi ile desteklenmiştir. GastronomGöz, el ile kalori kaydının işlemsel yükünü azaltarak obezite ve metabolik hastalıklarla mücadelede bilişim teknolojileri destekli bir çözüm önermektedir.

\vspace{2cm}

\textbf{Anahtar Kelimeler:} Derin Öğrenme, Yemek Tanıma, Kalori Tahmini, Evrişimsel Sinir Ağları, Mobil Sağlık





%%%%%%%%%%%%%%%%%%%%%%%%%%%%%%%%%%%%%%%%%%%%%%%%%%%%%%%%%%%%%%%%%%%
\newpage
\begin{center}
\textbf{\large ABSTRACT}

\textbf{SMART NUTRITION TRACKING SYSTEM WITH MOBILE APP INTEGRATION}
\end{center}

In this study, GastronomGoz, a deep learning-based food recognition and calorie calculation system, has been developed. The system analyzes food photographs captured by users' mobile devices to perform automatic classification, portion weight estimation, and nutritional value calculation. The ResNet50 convolutional neural network architecture for food classification, the U²-Net model for image segmentation, and the MiDaS algorithm for depth information support were employed. The backend layer was architected as a RESTful API using the Python Flask framework with 25 implemented endpoints, while the mobile application was fully developed cross-platform for iOS and Android using the Dart programming language and Flutter framework. The system incorporates JSON Web Token-based authentication, relational database management, real-time calorie tracking, daily and weekly statistics, meal editing and deletion, favorite marking, achievement badges, and user notification functionalities. Experimental results on Food-101 dataset demonstrated 73\% validation accuracy after 200 training epochs for food classification. The segmentation quality was found consistent with baseline studies. The weight estimation approach was designed using a portion-based and area-refined algorithm supported by MiDaS depth information. GastronomGoz presents an information technology-supported solution for combating obesity and metabolic diseases by reducing the operational burden of manual calorie tracking.

\vspace{2cm}

\textbf{Keywords:} Deep Learning, Food Recognition, Calorie Estimation, Convolutional Neural Networks, Mobile Health

\pagebreak{}
